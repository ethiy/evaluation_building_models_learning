\minitoc

\vfill

\clearpage

\section{\textsc{Scalabilitys analysis}}
    \label{sec::more_experiments::scalability}
    It is established that the scene composition can affect greatly model defect detection.
    This fact motivates studying training the classifier and testing prediction on different scenes.
    The goal is to prove the resilience of the prediction to unseen urban scenes.
    As the annotation process require a lot of effort, this trait is crucial to guarantee the scalability of this method.
    Different configurations are possible as depicted in Figure~\ref{fig::scalability_study}.
    In a first experiment, we train on one urban scene and test on another one (\textit{transferability} of the classifier model).
    In a second configuration, the classifier is trained on two scenes and tested on the last one: the goal is to investigate \textit{generalization}.
    The last experiment targets the \textit{representativeness} of a single 3-area dataset by trying multiple train-test splits.

    \begin{figure}[htbp]
        \ffigbox[\FBwidth]{
            \includestandalone[mode=buildnew, width=.55\textwidth]{figures/scalabitity_graph}
        }
        {
            \caption{\label{fig::scalability_study} A graph representing possible experiments: arrow origins represent training scenes while test ones are depicted as targets. \(Z_i, i=1,2,3\) represent the urban zones. All these nodes are assembled in one, meaning that all urban scenes were aggregated in on train/test node. The numbers indicate in which section each experiment is analyzed.}
        }
    \end{figure}

    \subsection{\textsc{Transferability study}}
        \subsubsection{\textsc{}}
        \subsubsection{\textsc{}}
        \subsubsection{\textsc{}}
    \subsection{\textsc{Generalization study}}
        \subsubsection{\textsc{}}
        \subsubsection{\textsc{}}
        \subsubsection{\textsc{}}
    \subsection{\textsc{Representativeness study}}
        \subsubsection{\textsc{}}

\section{\textsc{\texttt{Finesse} study}}
    \subsection{\textsc{Error family detection}}
    \subsection{\textsc{Detection of erroneous models}}

\section{\textsc{Richer features contributions}}
    \subsection{\textsc{Results}}
    \subsection{\textsc{Comparisons}}
