\minitoc

\phantom{M}

The goal of this introduction is to acquaint the reader with concepts that are manipulated in this study.
In section~\ref{sec::introduction::urban_3d_reconstruction}, we illustrate what is meant by urban \index{urban!\gls{acr::3d}!model}\gls{acr::3d} models, specifically, \index{building!model} building \gls{acr::3d} models.
The role of this chapter is to also motivate the need for a semantic \index{building!model!evaluation}evaluation of such models.
Indeed, in section~\ref{sec::introduction::building_model_evaluation}, we address different aspects of building \index{urban!\gls{acr::3d}!model}\gls{acr::3d} model inspection and the issues it raises.
We conclude by stating our contributions (cf. section~\ref{sec::introduction::contributions}) and the structure of the thesis (cf. section~\ref{sec::introduction::structure_of_thesis}).

\section{Urban 3D reconstruction}
    \label{sec::introduction::urban_3d_reconstruction}
    \index{\gls{acr::3d}!reconstruction}\gls{acr::3d} reconstruction is a wide research field that interests \index{photogrammetry}Photogrammetry, \index{computer vision}Computer Vision and \index{computer graphics}Computer Graphics communities.
    In summary, the main idea is, using some input sensor data (\gls{acr::lidar} point clouds or stereo images), to acquire the \gls{acr::3d} surface that bounds a studied object.
    In \index{\gls{acr::gis}}\gls{acr::gis}, \gls{acr::3d} data is instrumental to model the Earth surface at different scales.
    In particular, the field takes a special interest in \index{urban!area}urban areas.
    Objects present in such scenes obey usually some specific rules that constrain their shape.
    In this section, we dicuss applications of \index{urban!\gls{acr::3d}!model}urban \gls{acr::3d} models (cf. subsection~\ref{subsec::introduction::urban_3d_reconstruction::applications}).
    We tackle, afterwards, the \index{building!\gls{acr::3d}!model} building \gls{acr::3d} modeling subject (cf. subsection~\ref{subsec::introduction::urban_3d_reconstruction::building_3d_modeling}).
    This section ends, in subsection ~\ref{subsec::introduction::urban_3d_reconstruction::challenges}, with a listing of some important challenges that are still to be overcome.

    \subsection{Applications of urban \acrshort{acr::3d} models}
        \label{subsec::introduction::urban_3d_reconstruction::applications}
        In what follows, we present a brief survey of \index{urban!model!application} urban \gls{acr::3d} models applications.
        A more comprehensive study was presented in~\cite{ijgi4042842}.
        The goal is to persuade the reader of the relevance of urban \index{urban!\gls{acr::3d}!model}\gls{acr::3d} modeling and how much concerned is everybody, in a way or another, by this subject.
        \subsubsection{Urban planing related challenges}
        Urbanization raises serious issues that need to be adressed.
        Consequently, decision makers, and all stakeholders in general, need to have at their disposal adequate tools.
        \index{\gls{acr::3d}!model}\gls{acr::3d} models are, in this sense, suitable as shown hereafter.\\

        Cadastral data, since conception, came in the form of \index{\gls{acr::2d}!maps}\gls{acr::2d} maps\addref.
        Although many urban planning norms were always formulated taking into account height information \addref, the need for a \index{\gls{acr::3d}!cadastre}\gls{acr::3d} cadastral management was made relevant with the advent of complex architectural features~\cite{ijgi4042842}.
        To illustrate this case, we show in figure~\ref{fig::3d_cadastre_need_example}.
        \missingfigure[figwidth=6cm]{\label{fig::3d_cadastre_need_example}Testing a long text string}
        \begin{itemize}
            \item Cadastral data management: 3ddelft
            \item Urban planning simulation from cadastral data: brassebin, simplu
            \item Vehicule flow description: VarCity
            \item Wave propagation simulation : telecomunication companies
            \item traffic noise simulation
            \item Public concertation and decision making
            \item flood simulation for insurers owners decision makers
            \item fire hazards and propogation
        \end{itemize}
        \subsubsection{Sustainable developpement related challenges}
        \begin{itemize}
            \item Energy consumption simulation
            \item solar exposure
            \item microclimate simulation (jakarto quebec city, sylvie daniel)
        \end{itemize}
        \subsubsection{Autonomous navigation related challenges}
        \begin{itemize}
            \item Outdoor Localisation (Le Petit: osm + extrusion)
            \item civilian space inteligent navigation (alahi)
            \item industrial assistance
        \end{itemize}
        \subsubsection{Entertainement related challenges}
        \begin{itemize}
            \item realistic games (Call of duty: Paris)
            \item Tourism (VarCity), virtual reality
            \item Marketing -> Amine
        \end{itemize}
        \subsubsection{Security related challenges}
        \begin{itemize}
            \item Emergency planning
            \item (Para)Military intervention simulation
            \item Marketing: Grand Paris
        \end{itemize}
    \subsection{Building 3D modeling}
        \label{subsec::introduction::urban_3d_reconstruction::building_3d_modeling}
        Scale of reconstruction:
        \begin{itemize}
            \item Small scale: BIM/CAD models. Manually constructed. Very detailed but very expensive. Problems:
            \begin{itemize}
                \item Historical landmarks, old buildings: not always well documented: ex: Notre Dame de Paris.
                \item New construction project: reality is not always what is planified.
                \item Bim -> Geo: geometry is bad: self-intersection IFC2cityGML (Jantien stoter, hugo ledoux).
            \end{itemize}
            \item Large scale: GIS. Acquire urban scene geometry from sensor data. or Crowdsourcing: sketchup + building plans.
        \end{itemize}
        Objects highly dynamic:
        \begin{itemize}
            \item frequntly: city furniture, transportation vehicules, grass \& trees, water bodies (artificial lakes ...), pedestrians.
            \item not so much: ground, roads, bridges and buildings. Roads and ground are easier and Buildings are generally not. + sizable land cover. => important to model.\\
        \end{itemize}
        Importance of semantics. Not only geometric accuracy. Need to know building part architectural function. 3D model vs 3D mesh.
        semantics => discrete lod. cityGML LODS.\\
    \subsection{Obstacles in Building 3D modeling}
        \label{subsec::introduction::urban_3d_reconstruction::challenges}
        Widely studied in more than 20 years.\\
        \begin{itemize}
            \item Automatizing modeling. Needs manual corrections in high lods. Indoor open + facade parsing still active.
            \item trade-off fidelity vs compactness/semantics.
            \item Change detection in models.
            \item Crowdsourcing.
            \item Scale-space exploration: F Lafarge(JURSE) G Toya\& Marion dumont.
            \item Robustness to input noise.
        \end{itemize}
\section{Evaluating building models}
    \label{sec::introduction::building_model_evaluation}
    \subsection{Geometric consistancy inspection}
        Hugo ledoux, Gisquière. LoD detection.
    \subsection{Manual inspection}
        Once valid geometry. How good is the model? how close to reality?
        IGN case.
        expensive in man power. Not scalable.
    \subsection{Automatic evalutation}
        No one does. Only global error metrics. No semantics. No error localisation.
\section{Contributions}
    \label{sec::introduction::contributions}
    \subsection{Positioning}
        Structure properties => fidelity metrics insufficient
        No geometric issues studied here.
        Semantic evaluation: categorization of errors.
        Errors independent from: LoD, modeling method and urban scene.
    \subsection{Potential use}
        Paper.
    \subsection{Main contributions}
        Paper.
\section{Structure of the Thesis}
    \label{sec::introduction::structure_of_thesis}
