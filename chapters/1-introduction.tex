\minitoc

The goal of this introduction is to acquaint the reader with concepts that are manipulated in this study.
In fact, in section~\ref{sec::introduction::urban_3d_reconstruction}, we discuss what is meant by urban \index{model!\gls{acr::3d}}\gls{acr::3d} models, specifically, \index{model!building} building \gls{acr::3d} models.
The role of this chapter is also to motivate the need for a semantic \index{model!building!evaluation}evaluation of such models.
Indeed, in section~\ref{sec::introduction::building_model_evaluation}, we address different aspects of building model inspection and how difficult it can be.
We conclude by stating our contributions (cf. section~\ref{sec::introduction::contributions}) and the structure of the thesis (cf. section~\ref{sec::introduction::structure_of_thesis}).

\section{Urban 3D reconstruction}
    \label{sec::introduction::urban_3d_reconstruction}
    \index{reconstruction!\gls{acr::3d}}\gls{acr::3d} reconstruction is a wide research subject in Photogrammetry, Computer Vision and Compute Graphics communities.
    \begin{itemize}
        \item Idea: take sensor data (LiDAR point cloud or stereo images) procude 3d surface of objects.
        \item In GIS, interested mainly in urban object modeling. => more constraints on nature of objects.
        \item Start by Applications, then building modeling and finally challenges.
    \end{itemize} 
    \subsection{Applications of urban 3D models}
        Many application cases:
        \paragraph{Urban challenges}
        \begin{itemize}
            \item Cadastral data management: 3ddelft
            \item Urban planning simulation from cadastral data: brassebin, simplu
            \item Vehicule flow description: VarCity
            \item Wave propagation simulation : telecomunication companies
            \item traffic noise simulation
            \item Public concertation and decision making
            \item flood simulation for insurers owners decision makers
            \item fire hazards and propogation
        \end{itemize}
        \paragraph{Sustainable developpement}
        \begin{itemize}
            \item Energy consumption simulation
            \item solar exposure
            \item microclimate simulation (jakarto quebec city, sylvie daniel)
        \end{itemize}
        \paragraph{Autonomous navigation}
        \begin{itemize}
            \item Outdoor Localisation (Le Petit: osm + extrusion)
            \item civilian space inteligent navigation (alahi)
            \item industrial assistance
        \end{itemize}
        \paragraph{Entertainement industry}
        \begin{itemize}
            \item realistic games (Call of duty: Paris)
            \item Tourism (VarCity), virtual reality
            \item Marketing -> Amine
        \end{itemize}
        \paragraph{Security challenges}
        \begin{itemize}
            \item Emergency planning
            \item (Para)Military intervention simulation
            \item Marketing: Grand Paris
        \end{itemize}
    \subsection{Building 3D modeling}
        Scale of reconstruction:
        \begin{itemize}
            \item Small scale: BIM/CAD models. Manually constructed. Very detailed but very expensive. Problems:
            \begin{itemize}
                \item Historical landmarks, old buildings: not always well documented: ex: Notre Dame de Paris.
                \item New construction project: reality is not always what is planified.
                \item Bim -> Geo: geometry is bad: self-intersection IFC2cityGML (Jantien stoter, hugo ledoux).
            \end{itemize}
            \item Large scale: GIS. Acquire urban scene geometry from sensor data. or Crowdsourcing: sketchup + building plans.
        \end{itemize}
        Objects highly dynamic:
        \begin{itemize}
            \item frequntly: city furniture, transportation vehicules, grass \& trees, water bodies (artificial lakes ...), pedestrians.
            \item not so much: ground, roads, bridges and buildings. Roads and ground are easier and Buildings are generally not. + sizable land cover. => important to model.\\
        \end{itemize}
        Importance of semantics. Not only geometric accuracy. Need to know building part architectural function. 3D model vs 3D mesh.
        semantics => discrete lod. cityGML LODS.\\
    \subsection{Obstacles in Building 3D modeling}
        Widely studied in more than 20 years.\\
        \begin{itemize}
            \item Automatizing modeling. Needs manual corrections in high lods. Indoor open + facade parsing still active.
            \item trade-off fidelity vs compactness/semantics.
            \item Change detection in models.
            \item Crowdsourcing.
            \item Scale-space exploration: F Lafarge(JURSE) G Toya\& Marion dumont.
            \item Robustness to input noise.
        \end{itemize}
\section{Evaluating building models}
    \label{sec::introduction::building_model_evaluation}
    \subsection{Geometric consistancy inspection}
        Hugo ledoux, Gisquière. LoD detection.
    \subsection{Manual inspection}
        Once valid geometry. How good is the model? how close to reality?
        IGN case.
        expensive in man power. Not scalable.
    \subsection{Automatic evalutation}
        No one does. Only global error metrics. No semantics. No error localisation.
\section{Contributions}
    \label{sec::introduction::contributions}
    \subsection{Positioning}
        Structure properties => fidelity metrics insufficient
        No geometric issues studied here.
        Semantic evaluation: categorization of errors.
        Errors independent from: LoD, modeling method and urban scene.
    \subsection{Potential use}
        Paper.
    \subsection{Main contributions}
        Paper.
\section{Structure of the Thesis}
    \label{sec::introduction::structure_of_thesis}
