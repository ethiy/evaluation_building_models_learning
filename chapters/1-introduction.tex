\minitoc

% \phantom{M}
\vfill

The goal of this introduction is to acquaint the reader with concepts that are manipulated in this study.
In section~\ref{sec::introduction::urban_3d_reconstruction}, we illustrate what is meant by urban \index{urban!\gls{acr::3d}!model}\gls{acr::3d} models, and specifically, \index{building!model} building \gls{acr::3d} models.
The role of this chapter is to also motivate the need for a semantic \index{building!model!evaluation}evaluation of such models.
Indeed, in section~\ref{sec::introduction::building_model_evaluation}, we address different aspects of building \index{urban!\gls{acr::3d}!model}\gls{acr::3d} model inspection and the issues it raises.
We conclude by stating our contributions (cf. section~\ref{sec::introduction::contributions}) and the structure of the thesis (cf. section~\ref{sec::introduction::structure_of_thesis}).

\clearpage

\section{Urban \acrshort*{acr::3d} reconstruction}
    \label{sec::introduction::urban_3d_reconstruction}
    \index{\gls{acr::3d}!reconstruction}\gls{acr::3d} reconstruction is a wide research field that interests \index{photogrammetry}Photogrammetry, \index{computer vision}Computer Vision and \index{computer graphics}Computer Graphics communities.
    In summary, the main idea is, using some input sensor data (\gls{acr::lidar} point clouds or stereo images), to acquire the \gls{acr::3d} surface that bounds an object fo interest.
    In \index{\gls{acr::gis}}\gls{acr::gis}, \gls{acr::3d} data is instrumental to model the Earth surface at different scales.
    In particular, the field takes a special focus in \index{urban!area}urban areas.
    Objects present in such scenes obey usually some specific rules that constrain their shape.
    In this section, we dicuss applications of \index{urban!\gls{acr::3d}!model}urban \gls{acr::3d} models (cf. subsection~\ref{subsec::introduction::urban_3d_reconstruction::applications}).
    We tackle, afterwards, the field \index{building!\gls{acr::3d}!model} building \gls{acr::3d} modeling and the issues it raises(cf. subsection~\ref{subsec::introduction::urban_3d_reconstruction::building_3d_modeling}).
    This section ends, in subsection ~\ref{subsec::introduction::urban_3d_reconstruction::challenges}, with a listing of some important technical challenges that are still to be overcome in this domain.

    \subsection{Applications of urban \gls*{acr::3d} models}
        \label{subsec::introduction::urban_3d_reconstruction::applications}
        In what follows, we present a brief survey of \index{urban!model!application} urban \gls{acr::3d} models applications.
        A more comprehensive study was presented in~\textcite{ijgi4042842}.
        The goal is to persuade the reader of the relevance of urban \index{urban!\gls{acr::3d}!model}\gls{acr::3d} modeling and how much concerned can everybody be, in a way or another.

        \subsubsection{Urban planning related challenges}
            Urbanization raises some serious issues that need to be adressed.
            Consequently, decision makers, and all stakeholders in general, need to have at their disposal adequate tools.
            \index{\gls{acr::3d}!model}\gls{acr::3d} models are, in this sense, suitable as shown hereafter.\\
            Cadastral data, since conception, was presented in the form of \index{\gls{acr::2d}!maps}\gls{acr::2d} maps\addref.
            Although many urban planning norms were always formulated taking into account height, or depth, information \addref, the need for a \index{\gls{acr::3d}!cadastre}\gls{acr::3d} cadastral management was made relevant with the advent of complex architectural features~\textcite{ijgi4042842}.
            To illustrate this case, we show, in figure~\ref{fig::3d_cadastre_need_example}, how \gls{acr::2d} cadastral maps are insufficient when it comes to modeling overhangs.
            \todo{comment figure.}\\
            \begin{figure}[htpb]
                \centering
                \includegraphics[width=.7\textwidth]{example-image}            
                \caption{\label{fig::3d_cadastre_need_example} Example of a problematic situation where \gls{acr::3d} cadastre would be of help. Overhangs are hard to model in \gls{acr::2d}.}
            \end{figure}
            While \index{\gls{acr::3d}!cadastre}\gls{acr::3d} cadastre describes an urban scene at a certain time for fiscal purposes, urban plannars need the access to the same kind of information but in the future also.
            In fact, they need to know in advance the impact of urban norms they write on the evolution of their zone of interest. 
            One of the issues they study is \index{urban sprawl}urban sprawl~\parencite{ludlow2006urban}.
            The goal is to limit, as far as possible, \index{land occupation}land occupation~\parencite{TANNIER2012128} and predict the shape of the urban scene~\parencite{brasebin20183d} by tuning local urban plans.
            This can be acheived, for instance, through simulation, at different scales, based on a formal description of urban planning norms as shown by the work of~\textcite{Colomb17a}.
            In reality, the \gls{acr::2d} city footprint is not sufficient to assess its capacity to contain people.
            One has to compute the number of habitable units, which depends on the height of buildings.
            This motivates the need of \gls{acr::3d} data as shown in Figure~\ref{fig::3d_simulation}.
            The \index{vizualation} vizualation and \index{simulation}simulation of urban zones can be equally helpfull for \index{public consultation}public consultation~\parencite{WU2010291}.\\
            \begin{figure}[htpb]
                \centering
                \includegraphics[width=.7\textwidth]{example-image}            
                \caption{
                    \label{fig::3d_simulation} Example of \gls{acr::3d} urban scene simulated from known urban norms.
                    Urban planners can assess the projected urban density of future districts for better decision making.
                }
            \end{figure}
            \begin{figure}[htpb]
                \centering
                \includegraphics[width=.7\textwidth]{example-image}            
                \caption{
                    \label{fig::public_consultation} Example of public consultation tools using urban \gls{acr::3d} models.
                }
            \end{figure}
            Related to urban planning, \gls{acr::3d} models could be used as a reference for other planners.
            It could be used to describe the flow of vehicles and pedestrians in an urban environement as illustrated in~\textcite{Vanhoey:2017:VVS:3084363.3085085}.
            In addition, it could be used in physical simulations for urban applications.
            For instance, Communication companies need to have a \gls{acr::3d} model of urban scenes to predict signal propagation in the goal of optimal network planning~\parencite{yun2007radio}.
            It can be useful also for flood simulation.
            Predicting the height of overflowing water requires inherently a \gls{acr::3d} information.
            This is the usecase of~\textcite{varduhn2015multi}, where \gls{acr::3d} models are used to assess the flood risk.
            Such information could be instrumental for evacuation planning or for insurance managers.
            In the same direction, one can simulate fire propagation~\parencite{dimitropoulos2010fire} or estimating noise propagation in urban scenes~\parencite{stoter20083d} (\textit{cf.} Figure~\ref{fig::noise_propogation}).
            \begin{figure}[htpb]
                \centering
                \includegraphics[width=.7\textwidth]{example-image}            
                \caption{
                    \label{fig::noise_propogation} Example of noise propagation simulation using \gls{acr::3d} city models.
                }
            \end{figure}
            All these simulation derived informations could be fed to decision makers in order to better plan the cities of tomorrow.

        \subsubsection{Environemental challenges}
            Environement preservation is a all the more important in the coming years.
            Urban settlement are one of the most biggest energy consumers.
            A more efficient energy utilization is necessary to sustain the frantic growth of urban areas.
            This motivates the need to quantify the energy consumption of urban settlements~\parencite{WATE20153372} or retrofitting costs~\parencite{previtali2014automatic}.
            ~\textcite{biljecki2015propagation} use also \gls{acr::3d} models of buildings in order to predict solar irradiation.
            In fact, solar potential estimation can be useful for assessing the benefits of expensive solar panels projects.\\
            \begin{figure}[htpb]
                \centering
                \includegraphics[width=.7\textwidth]{example-image}            
                \caption{
                    \label{fig::solar_potential} Solar potential estimation using building \gls{acr::3d} models.
                    The slope, oriantation and dimensions of roofs are determinant factors in computing solar irradiation.
                }
            \end{figure}
            Another big environemental issue that affects cities is air pollution.
            Indeed, it has serious implications on human health as demonstrated in~\textcite{pascal2013assessing} and~\textcite{chen2013evidence}.
            In order to understand its dynamics, researchers simulate the local air flow (\textit{i.e.} the city microclimate) using computational fluid dynamics.
            This requires a detailed knowledge of the scenes layout.
            One way to acquire this information is through \index{urban!\gls{acr::3d}!model}\gls{acr::3d} models of urban settlements~\parencite{ujang2013unified}.
       
        \subsubsection{Autonomous navigation related challenges}
            Autonomous navigation has seen a great technological leap in recent years.
            Localization is an important step in visual navigation~\parencite{bonin2008visual}.
            \Gls{acr::3d} models play an important role in visual localization~\parencite{piasco2018survey, ijgi4042842}.\\
            The basic idea is to match an image with a (textured or not) with a known \gls{acr::3d} model of the city~\parencite{arth2015instant, ardeshir2014gis, cham2010estimating, christie2016semantics} as shown in Figure~\ref{fig::navigation}.
            \begin{figure}[htpb]
                \centering
                \includegraphics[width=.7\textwidth]{example-image}            
                \caption{
                    \label{fig::navigation} Position estimation using on \gls{acr::2d5} models of urban scenes.
                    This is possible through image pairing with these models.
                }
            \end{figure}
            Once the image matched, one can retrieve an absolute 6-\gls{acr::dog} pose estimation.
            This is especially helpfull in urban canyons as shown in~\textcite{piasco2018survey}.\\
            This can be applied also for indoor environements, such as social cues aware robots navigating alongside humans~\parencite{gupta2018social} or industrial grade robots\addref.
       
        \subsubsection{Entertainement related challenges}
            \gls{acr::3d} models appeals also to various agents in the entertainement industry.
            One of the first examples that comes to mind is the video games community~\parencite{watson2008procedural}.
            In their seek of reality, in order to engage the most customers as possible, studios reproduce entire cities as a virtual playing ground for the game story.
            We can cite the "Spider-Man" virtual New-York city\footnote{
                \href{https://www.polygon.com/2013/9/25/4702318/under-the-hood-of-infamous-second-son-hyper-real-seattle}{Under the hood of Infamous: Second Son's hyper-real Seattle}
            }\footnote{
                \href{https://www.polygon.com/e3/2018/6/12/17453588/spider-man-ps4-new-york-city-avengers-demo-preview}{How Spider-Man PS4’s New York City compares to the real thing}
            } or the realistic facsimile of Seattle in "Infamous Second Son"\footnote{
                \href{http://www.businessinsider.fr/us/spider-man-ps4-new-york-city-2018-9}{I'm blown away by the virtual New York City of 'Spider-Man' on PlayStation 4 — here's how it compares to the real thing}
            } as instances of such use.\\
            Tourism can also benefit from such models.
            In fact, virtual touring has become more attainable with works like~\textcite{koutsoudis20073d}.
            For instance, tourists can use it to prepare their trip by familiarizing themselves with the city they are visiting.
            This can be possible through mixed or augmented reality as shown by~\textcite{devaux20183d}.
            It can also be employed in the service of art.
            Actually,~\textcite{aubry2014painting} and~\textcite{russell2011automatic} illustrated how it is possible to align paintings or photographs with \gls{acr::3d} scenes.\\
            \begin{figure}[htpb]
                \centering
                \includegraphics[width=.7\textwidth]{example-image}            
                \caption{
                    \label{fig::augemented_reality} Alex devaux.
                }
            \end{figure}
            This last work could be also used to help marketers sell living units.
            Indeed, customers can, for example, visit a digitally reconstructed appartement and virtually furnish it~\parencite{kim2019planar}.
            Another application is living unit pricing.
            In fact, one would not have to travel to the asset location in order to assess it.
            Markerters can do so using its \gls{acr::3d} model.
            For instance, one of the determining factors in estimating buildings is the fa\c{c}ade visibility.
            The latter can be simply measured using building models, as shown by~\textcite{albrecht2013assessing}.

        \subsubsection{Security related challenges}
            Security and emergency fields are not the exception when it comes to the utilization of city \gls{acr::3d} models~\parencite{kwan2005emergency, ruppel2011designing}.
            For instance,~\textcite{chen2014application} shows how these models could be used for ladder trucks optimal deployement planning by firefighters.
            It can also be used to determine safe margins in the case of bomb disposal operations.
            This can be possible through explosion simulation in the urban environement of interest~\parencite{willenborg2015simulation}.\\
            \begin{figure}[htpb]
                \centering
                \includegraphics[width=.7\textwidth]{example-image}            
                \caption{
                    \label{fig::explosion_simulation} Explosion simulation in urban environements.
                }
            \end{figure}
            Security forces can equally benefit from the same models.
            \gls{acr::3d} models can be used to help analysing crime scenes~\parencite{wolff2009towards}.
            It can also be helpfull for crime prevention as proved by~\textcite{wolff2008geospatial}.
            These models could be instrumental in military applications~\textcite{zlatanova2002trends, budroni2010automatic}.
            Military forces could, indeed, use building \gls{acr::3d} model based augmented reality to train for intervention scenarii, such as hostage rescue operations.

    \subsection{Building \gls*{acr::3d} modeling}
        \label{subsec::introduction::urban_3d_reconstruction::building_3d_modeling}
        We have seen, previously, how city \gls{acr::3d} models can be instrumental.
        They have a large range of applications in entertainement, industry, security, urbanization and sustainable developpement.
        In this work, th focus is on outdoor, rather than indoor, modeling.\\
        We will raise, herein, the issue of outdoor city modeling.
        We will see how building reconstruction has a prominent role in the field.
        Afterwhat, we will discuss different building model acquisition techniques and how it influences their quality.
        We end with an examination of semantics in building models and the various used modeling strategies.

        \subsubsection{Outdoor city modeling}
            Urban environements are temporally dynamic in nature~\parencite{Vanhoey:2017:VVS:3084363.3085085}.
            However, constituent items do not evolve with uniform speed.
            We can distinguish, in fact, urban objects depending on their change rate.
            Pedestrians --- as well as all living animals in general --- and transportation vehicles are in perpetual movement.
            Water bodies and vegetation, in urban scenes, evolve with an annual or seasonal period.
            Last comes city furniture, roads, bridges, buildings and terrain which have a much lower change frequency.
            We will study, herein, how each of the three groups is modeled in \gls{acr::3d}.\\

            Besides technical difficulties, there are ethical and legal issues when reconstructing \gls{acr::3d} models of humans and vehicles.
            Indeed, accurate reconstruction involves person indentification.
            This has proven to be an intricate subject, as proved by~\textcite{tavani2011ethics} and~\textcite{thornton2010individual}.
            Adding to the previous discussion about the high temporal frequency of such objects, seeking the most faithfull models proves to be superfluous.
            In fact, one can populate city models by generic \gls{acr::cad} models of these humans~\parencite{shao2007autonomous} and vehicles.
            Even more so, a lot of aspects of human/vehicle and city interactions do not require \gls{acr::3d} modeling.
            For instance,~\textcite{lovaas1994modeling} can simulate pedestrian traffic flow, which is inherently a \gls{acr::2d} problem\footnote{
                We can safely assume that human do not fly in 2019.
            }, without requiring \gls{acr::3d} human models.\\
            There is little or no work, as far as we know, that is interested in water body \gls{acr::3d} modeling in urban areas.
            Vegetation can be modeled in details using \gls{acr::lidar} acquisition~\parencite{omasa20063d}.
            It is, however, too demanding in ressources and pointless in a large scale context.
            Trees are usually modeled by template matching, such as the ellipsoidal form model in~\textcite{lafarge_ijcv12}, or by generic, species dependent, \gls{acr::cad} models like in~\textcite{iovan2008detection}.\\

            In conclusion, regarding the first and second groups, was proven, the lack of motivation for precise \gls{acr::3d} reconstruction, in a large scale setting.
            Being less volatile through time, precisely modeling items from the third group seems to be easier.
            In fact, terrain relief can be modeled simply from a \gls{acr::dsm} or a \gls{acr::dtm}.
            Although not being so easy to detect~\parencite{mnih2010learning}, roads could be naturally modeled using simple planar structures.
            On the other hand, city furniture, bridges and buildings are more complex.
            While detailed accurate models are needed for buildings and bridges, they are not necessary for city furniture.
            Indeed, it is, for instance, pointless to model each single traffic signal.
            One would only need to detect its class and reconstruct it, accordingly, using a \gls{acr::cad} model of that same class.\\
            Out of the the last group only buildings and bridges pose problems.
            Actually, we can rule out the latter owing to, this time, a spatial frequency categorization.
            In terms of land cover, the most occuring objects in an urban environement are roads, vegetation and buildings \addref.
            Hence, modeling these three objects becomes vital in order to obtain a viable urban \gls{acr::3d} model.
            We have seen previously how roads and vegetation could be satisfactorily reconstructed using relatively simple models.
            This is, however, not the case of buildings.
            That is why, out of all urban features, buildings seem to attract the most attention in urban \gls{acr::3d} modeling.

        \subsubsection{Building \gls*{acr::3d} model types}
            Building modeling could be divided in two classes depending on the reconstruction scale (\textit{cf.} Figure~\ref{fig::bim_vs_gis}).
            At a small scale, \gls{acr::bim} or \gls{acr::cad} models could be used.
            On the contrary, \gls{acr::gis} models are more suitable at a large scale.\\

            \gls{acr::bim} models are volumetric in nature:
            Each element is represented by a volumetric primitive.
            These models are bottom-up.
            A \gls{acr::bim} model is manually constructed as a blueprints of a building before being constructed.
            It is then supposed to follow the buildings evolution in time until its destruction.
            This is the most detailed possible virtual representation of a building.\\
            However, it does not come without its own issues.
            First, we can see that it rules out all buildings that preceded the technology, especially, historical buildings.
            Secondly, since this type of models require a high interaction with experts, in order to follow the state of the real buildings, it would be almost impossible the fact that the model does not diverge from the reality\addref.
            Last but not least, are the geometric issues, like self-intersetions and non 2-manifoldness, that these models display as most \gls{acr::bim} tools do not perform geometric sanity checks.\\

            \begin{figure}[htpb]
                \centering
                \includegraphics[width=.7\textwidth]{example-image}            
                \caption{
                    \label{fig::bim_vs_gis} BIM vs GIS.
                }
            \end{figure}

            In contrast, \gls{acr::gis} models represent rather the surface of buildings.
            The goal is to describe the building geometry, as well as all urban itmes, at a large scale.
            Furthermore, this model type carries semantics related to all urban objects in the scene as well as their relations.
            These models could be acquired manually, as with \gls{acr::bim}, like the example of~\textcite{ref3dnat}.
            Another way to proceed consists in acquiring the geometry, automatically or interactively, using sensor data~\parencite{musialski2013survey}.
            Crowdsourcing could be equally used for large scale reconstruction of buildings as depicted in~\textcite{uden2013open}.\\

            Some works tackled the issue of bridging the two model types~\parencite{deng2016mapping}.
            This field is far from being fully mature as proved in~\textcite{stoter2018geo}.
            There are, in addition to the geometric inconsistancies that emanate from \gls{acr::bim} and \gls{acr::gis} model mapping,~\textcite{stoter2018geo} show how semantic ambiguities thwart the automatic conversion from one format to the other.\\

            In this work, the scalability of \gls{acr::3d} modeling is a main concern.
            As a consequence, hereafter, a special focus is given to automatic and, in a minor degree, interactive building \gls{acr::3d} modeling techniques.

        \subsubsection{Building \gls*{acr::3d} models meets semantics}
            The geometric accuracy of building models is not sufficient~\parencite{biljecki2016improved}.
            Semantics are needed, for instance, to identify the function of each architectural feature.
            In fact, the latter are usually composed of simple geometric shapes, mostly planar and usually single.
            Hence, a dense \gls{acr::3d} mesh is not necessarily the most geometrically accurate.
            Semantics imply, therefore, a need for compacity in the geometric representation of buildings.
            That is why it was used, for example, in addition to the \gls{acr::rmse}, as an evaluation metric in~\textcite{lafarge_ijcv12}.
            As a consequence, we distinguish, from now on, between building \gls{acr::3d} models and building \gls{acr::3d} meshes.
            While the latter accounts for the geometric precision, the other conveys semantic properties of the model.
            This is illustrated in Figure~\ref{fig::3dmodel_vs_3dmesh}.\\

            \begin{figure}[htpb]
                \centering
                \includegraphics[width=.7\textwidth]{example-image}            
                \caption{
                    \label{fig::3dmodel_vs_3dmesh} Example of the difference between a \gls{acr::3d} mesh compared to a \gls{acr::3d} model.
                    The first describes only the geometry, while the second is rich in semantics:
                    For each architectural feature corresponds a single geometric primitive.
                }
            \end{figure}

            Different strategies are used in building automatic \gls{acr::3d} modeling.
            Each method, balancing between compacity and geometric accuracy, targets a specific resolution.
            Ordered from the most to the less compact, these strategies are listed herein.
            In a Manhattan-world setting, one assumes that buildings are collections of boxes~\parencite{vanegas2010building, li2016manhattan}.
            ~\textcite{lafarge_ijcv12, nan2017polyfit} relied on the hypothesis that building are made of piecewise planar primitives.
            Richer grammars results in less compact but more accurate models, as is the case of~\textcite{demir2015procedural} or~\textcite{zeng2018neural}.
            Mesh simplification strategies comes last in terms of compacity~\parencite{verdie2015lod, zhou20102}.
            Figure~\ref{fig::modeling_strategies} summarises this comparison.\\

            \begin{figure}[htpb]
                \centering
                \includegraphics[width=.7\textwidth]{example-image}            
                \caption{
                    \label{fig::modeling_strategies} Modeling strategies and the targeted compacity and geometric accuracy.
                    Depending on the final use of the model, a compromise is chosen between the model compacity and its geometric accuracy.
                }
            \end{figure}

            Compacity, in building \gls{acr::3d} models, results from semantics.
            Consequently, it implies a discretization in resolution.
            Indeed,~\textcite{groger2012citygml} uses this semantic property to formalize a discrete intuitive \gls{acr::lod} scale.
            Even though the original \gls{acr::lod} specification were far from being mature it is widely used in the \gls{acr::gis} and Computer Vision communities~\parencite{biljecki2014formalisation, rau2006lod}.
            A detailed study of the issue was conducted in~\textcite{biljecki2014formalisation} and ~\textcite{biljecki2016improved}.
            All the same, this work will content itself with the simple intuitive definition of \glspl{acr::lod}.\\
            A \gls*{acr::lod}-0 model corresponds to the 2.5D footprint of the building.
            Next, the \gls*{acr::lod}-1 consists of the extrusion from the \gls*{acr::lod}-0 footprint up to a uniform height.
            \gls*{acr::lod}-2 enhances the previous model scale with more geometrically accurate roof structures.
            The \gls*{acr::lod}-3 reveals more details as it models small superstructures, as weel as openings.
            Last comes \gls*{acr::lod}-4 which holds indoor details that are ignored in this work.
            Figure~\ref{fig::lods} depicts these definitions.\\

            \begin{figure}[htpb]
                \centering
                \includegraphics[width=.7\textwidth]{example-image}            
                \caption{
                    \label{fig::lods} \gls{acr::lod} categorization used in this work.
                }
            \end{figure}
    \subsection{Obstacles in building \gls*{acr::3d} modeling}
        \label{subsec::introduction::urban_3d_reconstruction::challenges}
        The subject of building \gls{acr::3d} modeling has been widely studied in more than twenty years.
        Still, there are some unsolved issues in the field~\parencite{musialski2013survey, lafarge2015some}.
        Presented here is the same challenge classification as in~\textcite{lafarge2015some}.

        \subsubsection{Data acquisition}
            Sensor data acquisition can raise some serious issues, in signal processing in general, as in \gls{acr::3d} modeling in particular.
            In fact, noise is an integral part of physical measurement processes.
            One should take good care in avoiding error propagation through their processing pipelines.
            For instance, outliers in point-of-interest detection can render a photogrammetrically constructed mesh accumulate sizable geometric errors.\\
            Missing data is also a real issue in \gls{acr::3d}, as some background objects in the scene could be easily occluded by objects in the foreground.
            One way of dealing with this type of obstacles, is to multiply the data acquisition settings.
            This can, actually, help mitigate not only occlusion problems but also noise interference.\\
            However, too much data heterogenuity can also hinder the \gls{acr::3d} modeling process.
            In fact, with more accessible data acquired using different sensors (stereoimages, \gls{acr::radar} and \gls{acr::lidar}, for instance), in various settings (aerial, sattelite or terrestrial) and circumstances (rainy, sunny, foggy\dots and night-time, day-time\dots), other hurdles need to be overcome.
            For one, more data does not always mean more knowledge, as demonstrated in~\textcite{brachmann2018learning}.
            In fact, one should take good care in choosing how to fuse their input data.
            For instance, it needs be coregistered in the same referential~\parencite{mezian2016uncertainty}.
            There will be also some variability in radiometry, and all signal properties in general, that needs to be taken into account.
            \begin{figure}[htpb]
                \centering
                \includegraphics[width=.7\textwidth]{example-image}            
                \caption{
                    \label{fig::3d_aerial_terrestrial_fusion} Example of a fusion of aerial and terrestrial \gls{acr::lidar} point clouds for building modeling~\parencite{kedzierski2014terrestrial}.
                }
            \end{figure}

        \subsubsection{Modeling automatization}
            For a semantically aware and geometrically accurate representation of a building, it is better to choose an interactive process~\parencite{musialski2013survey}.\\
            In fact, automatic reconstruction algorithms rely on some \textit{a prioiri} properties that are not always true.
            Such hypotheses can come in grammar regularization & grammar of shapes & finite set of primitives.
            This does noit hold at large scales.
            In same city , country or continent, architectural styles change depend on location but also the history of th building and function\\
            As a result, some choose to alterne automatic and interactive.
            

\section{Evaluating building models}
    \label{sec::introduction::building_model_evaluation}
    \subsection{Geometric consistancy inspection}
        Hugo ledoux, Gisquière. LoD detection.
    \subsection{Manual inspection}
        Once valid geometry. How good is the model? how close to reality?
        IGN case.
        expensive in man power. Not scalable.
    \subsection{Automatic evalutation}
        No one does. Only global error metrics. No semantics. No error localisation.
\section{Contributions}
    \label{sec::introduction::contributions}
    \subsection{Positioning}
        Structure properties => fidelity metrics insufficient
        No geometric issues studied here.
        Semantic evaluation: categorization of errors.
        Errors independent from: LoD, modeling method and urban scene.
    \subsection{Potential use}
        Paper.
    \subsection{Main contributions}
        Paper.
\section{Structure of the Thesis}
    \label{sec::introduction::structure_of_thesis}
