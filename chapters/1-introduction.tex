\minitoc

\phantom{M}

The goal of this introduction is to acquaint the reader with concepts that are manipulated in this study.
In section~\ref{sec::introduction::urban_3d_reconstruction}, we illustrate what is meant by urban \index{urban!\gls{acr::3d}!model}\gls{acr::3d} models, and specifically, \index{building!model} building \gls{acr::3d} models.
The role of this chapter is to also motivate the need for a semantic \index{building!model!evaluation}evaluation of such models.
Indeed, in section~\ref{sec::introduction::building_model_evaluation}, we address different aspects of building \index{urban!\gls{acr::3d}!model}\gls{acr::3d} model inspection and the issues it raises.
We conclude by stating our contributions (cf. section~\ref{sec::introduction::contributions}) and the structure of the thesis (cf. section~\ref{sec::introduction::structure_of_thesis}).

\clearpage

\section{Urban \gls{acr::3d} reconstruction}
    \label{sec::introduction::urban_3d_reconstruction}
    \index{\gls{acr::3d}!reconstruction}\gls{acr::3d} reconstruction is a wide research field that interests \index{photogrammetry}Photogrammetry, \index{computer vision}Computer Vision and \index{computer graphics}Computer Graphics communities.
    In summary, the main idea is, using some input sensor data (\gls{acr::lidar} point clouds or stereo images), to acquire the \gls{acr::3d} surface that bounds an object fo interest.
    In \index{\gls{acr::gis}}\gls{acr::gis}, \gls{acr::3d} data is instrumental to model the Earth surface at different scales.
    In particular, the field takes a special focus in \index{urban!area}urban areas.
    Objects present in such scenes obey usually some specific rules that constrain their shape.
    In this section, we dicuss applications of \index{urban!\gls{acr::3d}!model}urban \gls{acr::3d} models (cf. subsection~\ref{subsec::introduction::urban_3d_reconstruction::applications}).
    We tackle, afterwards, the field \index{building!\gls{acr::3d}!model} building \gls{acr::3d} modeling and the issues it raises(cf. subsection~\ref{subsec::introduction::urban_3d_reconstruction::building_3d_modeling}).
    This section ends, in subsection ~\ref{subsec::introduction::urban_3d_reconstruction::challenges}, with a listing of some important technical challenges that are still to be overcome in this domain.

    \subsection{Applications of urban \acrshort{acr::3d} models}
        \label{subsec::introduction::urban_3d_reconstruction::applications}
        In what follows, we present a brief survey of \index{urban!model!application} urban \gls{acr::3d} models applications.
        A more comprehensive study was presented in~\textcite{ijgi4042842}.
        The goal is to persuade the reader of the relevance of urban \index{urban!\gls{acr::3d}!model}\gls{acr::3d} modeling and how much concerned can everybody be, in a way or another.

        \subsubsection{Urban planning related challenges}
        Urbanization raises some serious issues that need to be adressed.
        Consequently, decision makers, and all stakeholders in general, need to have at their disposal adequate tools.
        \index{\gls{acr::3d}!model}\gls{acr::3d} models are, in this sense, suitable as shown hereafter.\\

        Cadastral data, since conception, was presented in the form of \index{\gls{acr::2d}!maps}\gls{acr::2d} maps\addref.
        Although many urban planning norms were always formulated taking into account height, or depth, information \addref, the need for a \index{\gls{acr::3d}!cadastre}\gls{acr::3d} cadastral management was made relevant with the advent of complex architectural features~\textcite{ijgi4042842}.
        To illustrate this case, we show, in figure~\ref{fig::3d_cadastre_need_example}, how \gls{acr::2d} cadastral maps are insufficient when it comes to modeling overhangs.
        \todo{comment figure.}\\
        \begin{figure}[h]
            \centering
            \includegraphics[width=\textwidth]{example-image}             
            \caption{\label{fig::3d_cadastre_need_example} Example of a problematic situation where \gls{acr::3d} cadastre would be of help. Overhangs are hard to model in \gls{acr::2d}.}
        \end{figure}
        While \index{\gls{acr::3d}!cadastre}\gls{acr::3d} cadastre describes an urban scene at a certain time for fiscal purposes, urban plannars need the access to the same kind of information but in the future also.
        In fact, they need to know in advance the impact of urban norms they write on the evolution of their zone of interest.  
        One of the issues they study is \index{urban sprawl}urban sprawl~\parencite{ludlow2006urban}.
        The goal is to limit, as far as possible, \index{land occupation}land occupation~\parencite{TANNIER2012128} and predict the shape of the urban scene~\parencite{brasebin20183d} by tuning local urban plans.
        This can be acheived, for instance, through simulation, at different scales, based on a formal description of urban planning norms as shown by the work of~\textcite{Colomb17a}.
        In reality, the \gls{acr::2d} city footprint is not sufficient to assess its capacity to contain people.
        One has to compute the number of habitable units, which depends on the height of buildings.
        This motivates the need of \gls{acr::3d} data as shown in Figure~\ref{fig::3d_simulation}.
        The \index{vizualation} vizualation and \index{simulation}simulation of urban zones can be equally helpfull for \index{public consultation}public consultation~\parencite{WU2010291}.\\
        \begin{figure}[h]
            \centering
            \includegraphics[width=\textwidth]{example-image}             
            \caption{
                \label{fig::3d_simulation} Example of \gls{acr::3d} urban scene simulated from known urban norms.
                Urban planners can assess the projected urban density of future districts for better decision making.
            }
        \end{figure}
        \begin{figure}[h]
            \centering
            \includegraphics[width=\textwidth]{example-image}             
            \caption{
                \label{fig::public_consultation} Example of public consultation tools using urban \gls{acr::3d} models.
            }
        \end{figure}
        Related to urban planning, \gls{acr::3d} models could be used as a reference for other planners.
        It could be used to describe the flow of vehicles and pedestrians in an urban environement as illustrated in~\textcite{Vanhoey:2017:VVS:3084363.3085085}.
        In addition, it could be used in physical simulations for urban applications.
        For instance, Communication companies need to have a \gls{acr::3d} model of urban scenes to predict signal propagation in the goal of optimal network planning~\parencite{yun2007radio}.
        It can be useful also for flood simulation.
        Predicting the height of overflowing water requires inherently a \gls{acr::3d} information.
        This is the usecase of~\textcite{varduhn2015multi}, where \gls{acr::3d} models are used to assess the flood risk.
        Such information could be instrumental for evacuation planning or for insurance managers.
        In the same direction, one can simulate fire propagation~\parencite{dimitropoulos2010fire} or estimating noise propagation in urban scenes~\parencite{stoter20083d} (\textit{cf.} Figure~\ref{fig::noise_propogation}).
        \begin{figure}[h]
            \centering
            \includegraphics[width=\textwidth]{example-image}             
            \caption{
                \label{fig::noise_propogation} Example of noise propagation simulation using \gls{acr::3d} city models.
            }
        \end{figure}
        All these simulation derived informations could be fed to decision makers in order to better plan the cities of tomorrow.

        \subsubsection{Environemental challenges}
        Environement preservation is a all the more important in the coming years.
        Urban settlement are one of the most biggest energy consumers.
        A more efficient energy utilization is necessary to sustain the frantic growth of urban areas.
        This motivates the need to quantify the energy consumption of urban settlements~\parencite{WATE20153372} or retrofitting costs~\parencite{previtali2014automatic}.
        ~\textcite{biljecki2015propagation} uses also \gls{acr::3d} models of buildings in order to predict solar irradiation.
        In fact, solar potential estimation can be useful for assessing the benefits of expensive solar panels projects.\\
        \begin{figure}[h]
            \centering
            \includegraphics[width=\textwidth]{example-image}             
            \caption{
                \label{fig::solar_potential} Solar potential estimation using building \gls{acr::3d} models.
                The slope, oriantation and dimensions of roofs are determinant factors in computing solar irradiation.
            }
        \end{figure}
        Another big environemental issue that affects cities is pollution.
        \begin{itemize}
            \item microclimate simulation (jakarto quebec city, sylvie daniel)
        \end{itemize}
        \subsubsection{Autonomous navigation related challenges}
        \begin{itemize}
            \item Outdoor Localisation (Le Petit: osm + extrusion)
            \item civilian space inteligent navigation (alahi)
            \item industrial assistance
        \end{itemize}
        \subsubsection{Entertainement related challenges}
        \begin{itemize}
            \item realistic games (Call of duty: Paris)
            \item Tourism (VarCity), virtual reality
            \item Marketing -> Amine
        \end{itemize}
        \subsubsection{Security related challenges}
        \begin{itemize}
            \item Emergency planning
            \item (Para)Military intervention simulation
            \item Marketing: Grand Paris
        \end{itemize}
    \subsection{Building \gls{acr::3d} modeling}
        \label{subsec::introduction::urban_3d_reconstruction::building_3d_modeling}
        Scale of reconstruction:
        \begin{itemize}
            \item Small scale: BIM/CAD models. Manually constructed. Very detailed but very expensive. Problems:
            \begin{itemize}
                \item Historical landmarks, old buildings: not always well documented: ex: Notre Dame de Paris.
                \item New construction project: reality is not always what is planified.
                \item Bim -> Geo: geometry is bad: self-intersection IFC2cityGML (Jantien stoter, hugo ledoux).
            \end{itemize}
            \item Large scale: GIS. Acquire urban scene geometry from sensor data. or Crowdsourcing: sketchup + building plans.
        \end{itemize}
        Objects highly dynamic:
        \begin{itemize}
            \item frequntly: city furniture, transportation vehicules, grass \& trees, water bodies (artificial lakes ...), pedestrians.
            \item not so much: ground, roads, bridges and buildings. Roads and ground are easier and Buildings are generally not. + sizable land cover. => important to model.\\
        \end{itemize}
        Importance of semantics. Not only geometric accuracy. Need to know building part architectural function. \gls{acr::3d} model vs \gls{acr::3d} mesh.
        semantics => discrete lod. cityGML LODS.\\
    \subsection{Obstacles in Building \gls{acr::3d} modeling}
        \label{subsec::introduction::urban_3d_reconstruction::challenges}
        Widely studied in more than 20 years.\\
        \begin{itemize}
            \item Automatizing modeling. Needs manual corrections in high lods. Indoor open + facade parsing still active.
            \item trade-off fidelity vs compactness/semantics.
            \item Change detection in models.
            \item Crowdsourcing.
            \item Scale-space exploration: F Lafarge(JURSE) G Toya\& Marion dumont.
            \item Robustness to input noise.
        \end{itemize}
\section{Evaluating building models}
    \label{sec::introduction::building_model_evaluation}
    \subsection{Geometric consistancy inspection}
        Hugo ledoux, Gisquière. LoD detection.
    \subsection{Manual inspection}
        Once valid geometry. How good is the model? how close to reality?
        IGN case.
        expensive in man power. Not scalable.
    \subsection{Automatic evalutation}
        No one does. Only global error metrics. No semantics. No error localisation.
\section{Contributions}
    \label{sec::introduction::contributions}
    \subsection{Positioning}
        Structure properties => fidelity metrics insufficient
        No geometric issues studied here.
        Semantic evaluation: categorization of errors.
        Errors independent from: LoD, modeling method and urban scene.
    \subsection{Potential use}
        Paper.
    \subsection{Main contributions}
        Paper.
\section{Structure of the Thesis}
    \label{sec::introduction::structure_of_thesis}
