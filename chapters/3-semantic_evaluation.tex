\minitoc

\vfill

\clearpage

\section{\textsc{The general framework}}
    \label{sec::semantic_evaluation::general_framework}

    \subsection{\textsc{Hierarchization and modularity}}
        \label{subsec::semantic_evaluation::general_framework::hierarchization_moderularity}
        As stated previously in subsection~\ref{subsec::introduction::contributions::positioning}, a semantic evaluation implies a categorization of errors affecting building models.
        In the same subsection, we discussed properties that are desired so as to achieve a large scale and automatic semantic evaluation.
        Before delving into details, the implications of such properties on error definitions are examined.

        \subsubsection{\textsc{Generalizability \textit{vs.} exhaustivity compromise}}
            Two conditions are required in the error classification.
            The first one entails the error definitions incongruence with the evaluated urban scenes.
            This implies the proposed error categorization capacity to be generalizable.
            The second is the dissonance between the latter and the input evaluation models.
            This conveys the exhaustivity of the evaluation.\\
            The two notions are condradictory.
            At one hand, every possible error at any level should be taken into account, while, on the other, the categorization has to stay always relevant no matter the origin of those models.
            A compromise have to be reached in the definition of errors between these two properties.
        
        \subsubsection{\textsc{General structure}}
            We porpose a hierarchical structure of the categorization of errors in order to mitigate the last point.
            The more high in the ladder the error is, translates to more generalizability (and less exhaustivity) of the latter.
            At the same level, to avoid having to exhaustively list all possible errors, the identified defects are described modularly based on some predefined independent errors.
            This helps cover a wide range of possible defects while the basic errors are chosen to be as generalizable as possible.
            Hierarchization and modularity are the main ingredients in our proposed flexible framework.\\

            To implement these properties for the error taxonomy, we rely on two criterea for error compilation: the input building model \gls{acr::lod} and the error semantic precision level, named henceforth \texttt{finesse} (cf. Figure~\ref{fig::taxonomy}).
            Different degrees of \texttt{finesse} describe, from coarse to fine, the specificity of defects.
            It corresponds to the error hierarchy levels.
            The \gls{acr::lod} is used on the other hand to differenciate between errors in the same specificity level.
            Multiple errors at the same \texttt{finesse} can indeed affect the same building.
            For instance, topological defects almost always induce (and hence co-occur) with geometrical ones.
            Errors with maximal \texttt{finesse} are called \texttt{atomic} errors.
            \texttt{Atomic} errors are to be intuitively correlated to independent actions needed by an operator or an algorithm so as to correct the model.

    \subsection{\textsc{A general classification of errors}}
        \label{subsec::semantic_evaluation::general_framework::error_classification}
        Herein, based on the previous discussion, a general layout is detailed for building model evaluation.

        At a first level, model qualifiability is studied.
        In fact, aside from formatting issues or geometric inconsistencies~\parencite{ledoux2018val3dity}, other reasons make building models unqualifiable.
        For instance, buildings can be occluded by vegetation and thus cannot be assessed with most of the remote sensing data sources.
        Generally speaking, input models can be impaired by some pathological cases that are outside our evaluation framework.
        In consequence, \texttt{Qualifiable} models are distinguished here from \texttt{Unqualifiable} buildings.
        This first level corresponds to a \texttt{finesse} equal to 0. At the \texttt{finesse} level 1, we predict the correctness of all qualifiable buildings.
        It is the lowest semantization level at which the evaluation of a model is expressed.
        Then, a model is either \texttt{Valid} or \texttt{Erroneous}.
        Most state-of-the-art evaluation methods address errors up to this level.\\
        Model errors are grouped into two families depending on the underlying \gls{acr::lod}.
        The first family of errors \texttt{Building Errors} affects the building in its entirety.
        It corresponds to an accuracy evaluation at \gls{acr::lod}-0 (footprint errors) $\cup$ \gls{acr::lod}-1 (height/geometric error).
        At the next \gls{acr::lod}-2, the family \texttt{Facet Errors} gathers defects that can alter the facet accuracy of fa\c{c}ades or roofs (\gls{acr::lod}-2) as well as superstructures and openings (\gls{acr::lod}-3).\\
        Each family contains \texttt{atomic} errors of maximal \texttt{finesse} equal to 3.
        Although they can co-occur in the same building model and across different families, these errors are semantically independent.
        They represent specific topological or geometric defects.
        Topological errors translate inaccurate structural modeling, while geometric defects raise positioning infidelity.\\

        The general structure is not fixed and can evolve to adapt to more cases.
        In fact, instead of grouping \gls{acr::lod}-2 and \gls{acr::lod}-3 errors, the latter can be made into a different family that can be called \texttt{Superstructure Errors}.
        Due to the lack of sufficient observations, we did not make this choice in order to guaranty the generalizability of the taxonomy.
        Another alternative consists, for instance, in gathering error families by resolution: this will produce a continuum of errors families going from the coarsest level that would correspond to \texttt{Building Errors} to the finest possible one.
        This last option, although offering an exhaustive and potentially generalizable taxonomy, was ruled out since it does not provide a truly semantic description of the errors.
        % For example, an operator that is presented with the error ``segmentation error at resolution 20\%'' would not have any information on 
        Regarding \texttt{finesse}, it is also possible to have additional levels.
        The maximal level of 3 was chosen in order to preserve the generalizability of the taxonomy, since the more specific the error categorization is, the more observations we need to define the corresponding errors.

        \begin{figure}[h]
            \begin{center}
                \includestandalone[mode=buildnew, width=\textwidth]{figures/taxonomy_tree}
                \caption{
                    \label{fig::taxonomy} 
                    The proposed taxonomy structure.
                    In our case of very high resolution overhead image modeling, only two family errors are depicted.
                    At \texttt{finesse} level 2, hierarchization is possible: the \textbf{exclusivity} parameter can thus act.
                    However, it is not the case at the \texttt{atomic} errors level since they are independent.
                }
            \end{center}
        \end{figure}

\section{\textsc{Application to the geospatial overhead case}}
    \label{sec::semantic_evaluation::overhead}
    Our observations were based on large datasets of \gls{acr::3d} models of buildings reconstructed automatically using \gls{acr::vhr} geospatial images or, if available, \gls{acr::lidar} point clouds.
    We apply the framework introduced in the previous subsection to our special case.
    To do so, we will define the \texttt{atomic} errors before discussing their propoerties.

    \subsection{\textsc{Atomic error definitions}}
        \label{subsec::semantic_evaluation::overhead::atomic}
        In the template structure presented in section~\ref{subsec::semantic_evaluation::general_framework::error_classification}, were left out the \texttt{atomic} error definitions.
        Indeed, since they represent the most specific level, their choice is critical to guaranty both the exhaustivity and generalizability we are hoping for.
        We conducted a thourough inspection of all defects that we detected in our datasets and came up with the following definitions (\textit{cf.} Figures~\ref{fig::taxonomy}).
        Eventhough these errors were defined based on models of buildings using overhead acquired data at large scales, we think they are exhaustive enough to describe the quality of models in other settings (fa\c{c}ade modeling, manually plotted \gls{acr::3d} models).

        \subsubsection{\texttt{Building errors} \textsc{family}}
            Herein are presented the \texttt{atomic} errors regarding the \gls{acr::lod}-0 and \gls{acr::lod}-1 ascpects.

            \paragraph{\texttt{\acrlong*{acr::bus}}}
                \gls{acr::bus} corresponds to the case where two or more buildings are modeled as one.
                In Figure~\ref{fig::bus}, two distinct buildings were identified as one building, eventhough they can be visually distinguished.\\

                \begin{figure}[H]
                    \centering
                    \ffigbox[\FBwidth]{
                        \begin{subfloatrow}[2]
                            \ffigbox[\FBwidth]{
                                \includegraphics[width=.45\textwidth]{example-image}
                            }{
                                \label{subfig::bus_3d}
                                \caption{
                                    Depiction of the a case where two building that are reconstructed as one.
                                }
                            }
                            \ffigbox[\FBwidth]{
                                \includegraphics[width=.45\textwidth]{images/errors/building/under_segmentation}
                            }{
                                \label{subfig::bus_2d}
                                \caption{
                                    Nadir projection of the erroneous building superposed on the corresponding orthoimage.
                                    We can recognize, based on the color differences of roof tiles, the existance of two buildings instead of one.
                                }
                            }
                        \end{subfloatrow}
                    }{
                        \caption{
                            \label{fig::bus}
                            Illustration of a \gls{acr::bus} error.
                        }
                    }
                \end{figure}

                This is a very common error which results from a faulty footprint of the building.
                The latter is either retrieved automatically during the modeling~\parencite{lafarge2012creating}, or is provided as input~\parencite{durupt2006automatic}.
                The first case is the most error inducing as it relies on extrinsic large scale remote sensing data that are devoid of semantics.
                The second is expected to be more close to the reality, but can be unsuitable if outdated.

            \paragraph{\texttt{\acrlong*{acr::bos}}}
                \gls{acr::bos} corresponds to the case where one building is subdivided into two or more.
                This is the opposite of the previous situation.
                Figure~\ref{fig::bos} shows a single building that, when modelled, was subdivided into three parts.\\

                \begin{figure}[H]
                    \centering
                    \ffigbox[\FBwidth]{
                        \begin{subfloatrow}[2]
                            \ffigbox[\FBwidth]{
                                \includegraphics[width=.45\textwidth]{example-image}
                            }{
                                \label{subfig::bos_3d}
                                \caption{
                                    Depiction of the a case where two building that are reconstructed as one.
                                }
                            }
                            \ffigbox[\FBwidth]{
                                \includegraphics[width=.45\textwidth]{images/errors/building/over_segmentation}
                            }{
                                \label{subfig::bos_2d}
                                \caption{
                                    This a nadir projection of a single building that was modeled into three different ones depicted here in three different colors.
                                }
                            }
                        \end{subfloatrow}
                    }{
                        \caption{
                            \label{fig::bos}
                            Illustration of a \gls{acr::bos} error.
                        }
                    }
                \end{figure}

                This is also a very common error.
                It is the consequence of the same reasons as the under segmentation that was earlier discussed.
                Both these errors are highly semantic and ,thus, creates confusion between both classes.
                Depending on the chosen semantics, a building part (in the sense of CityGML) can be also considered as a single building in other cases.
                In fact, there is no single geometric characteristic that can help separate buildings, such as convexity or compactness.
                This issue is expected to weight negatively on the predictive capacity of the proposed evaluation approach as will be studied farther in sections~\ref{}.

            \paragraph{\texttt{\acrlong*{acr::bib}}}
                \gls{acr::bib} corresponds to the case where at least one building footprint border is incorrectly located.
                A sample is shown in Figure~\ref{fig::bib}.\\

                \begin{figure}[H]
                    \centering
                    \ffigbox[\FBwidth]{
                        \begin{subfloatrow}[2]
                            \ffigbox[\FBwidth]{
                                \includegraphics[width=.45\textwidth]{example-image}
                            }{
                                \label{subfig::bib_3d}
                                \caption{\gls{acr::3d} depiction of the error.}
                            }
                            \ffigbox[\FBwidth]{
                                \includegraphics[width=.45\textwidth]{images/errors/building/border}
                            }{
                                \label{subfig::bib_2d}
                                \caption{
                                    In red is the reconstructed model border that is far from being close to reality as can be checked using the orthoimage.
                                    We can distinguish in green the actual edge using a Nadir projection.
                                }
                            }
                        \end{subfloatrow}
                    }{
                        \caption{
                            \label{fig::bib}
                            Illustration of a \gls{acr::bib} error.
                        }
                    }
                \end{figure}

                This is a purely geometric error that is caused by an imprecise footprint.
                Semantics intervene as \gls{acr::bib} is mainly linked to the end user preferrences: one can ignore errors up to a certain threshold. 
                The footprint border precision is understandably dependent on the used input data during the modeling process.
                It is also expected that the error detection precision will hinge on the resolution of the used reference data and its registration accuracy.
                Regarding automatic modeling methods, border imprecision can be attributed to the quality of the used edge detection algorithms~\parencite{nan2015template,baillard1999automatic,werner2002new} or inaccurate surface estimation\footnote{
                    The border is computed as intersection of the detected surfaces.
                }~\parencite{durupt2006automatic,xiong2014graph}.
                Outside the scope this study, one can try also to estimate the imprecision so as to correct the reconstruction.

            \paragraph{\texttt{\acrlong*{acr::bit}}}
                \gls{acr::bit} corresponds to the case where the building footprint suffers from topological defects.
                Modeled as a \gls{acr::2d} flat surface, the cases that fall into this label are:
                \begin{itemize}
                    \item missing inner court: corresponds to a missing hole;
                    \item inaccurate border shape:  is due to a wrong primitive fitting: the shape of the footprint can be better described by a diffenrent geometrical shape.
                            In Figure~\ref{fig::bit} is given an example where the polygon has a wrong number of sides.
                            Another case is where a circular footprint can be approximated by a polygon.
                \end{itemize}

                \begin{figure}[H]
                    \centering
                    \ffigbox[\FBwidth]{
                        \begin{subfloatrow}[2]
                            \ffigbox[\FBwidth]{
                                \includegraphics[width=.45\textwidth]{example-image}
                            }{
                                \label{subfig::bit_3d}
                                \caption{\gls{acr::3d} depiction of the error.}
                            }
                            \ffigbox[\FBwidth]{
                                \includegraphics[width=.45\textwidth]{images/errors/building/topology}
                            }{
                                \label{subfig::bit_2d}
                                \caption{
                                    The nadir projection comparison with the corresponding orthoimage gives away (in green) the correct footprint shape compared to the one that was reconstructed (in red).
                                }
                            }
                        \end{subfloatrow}
                    }{
                        \caption{
                            \label{fig::bit}
                            Illustration of a \gls{acr::bit} error.
                        }
                    }
                \end{figure}

                This error, as the earlier one, is a result of a defective edge estimation.
                The main difference between the two is the fact that \gls{acr::bib} is geometric in nature while \gls{acr::bit} is topological.
                Both errors are independent and can overlap.

            \paragraph{\texttt{\acrlong*{acr::big}}}
                \gls{acr::big} corresponds to the case of inaccurate building geometric estimation.
                Up to \gls{acr::lod}-1, it coincides with height imprecision.
                This is yet again a geometric error.
                Semantics play a role in the definition of the height of a building.
                It can be defined as the height at the highest point, the mediane height or any other valid alternative.
                In case of evaluating at higher than \gls{acr::lod}-2, this error is not reported as it becomes redundant with below delineated errors.
                In fact, if a geometric error is detected at the facet level then it will naturally impact negatively on the geometry of the model as a whole.
            
        \subsubsection{\texttt{Facet errors} \textsc{family}}
            In this sub-subsection, \gls{acr::lod}-2 and \gls{acr::lod}-3 corresponding \texttt{atomic} errors are presented.

            \paragraph{\texttt{\acrlong*{acr::fus}}}
                \gls{acr::fus} corresponds to the case where one facet is subdivided into two or more facets.
                This is the same kind of error as \gls{acr::bus} but at the facet level.\\

                \begin{figure}[H]
                    \centering
                    \ffigbox[\FBwidth]{
                        \begin{subfloatrow}[2]
                            \ffigbox[\FBwidth]{
                                \includegraphics[width=.45\textwidth]{example-image}
                            }{
                                \label{subfig::fus_3d}
                                \caption{
                                    Depiction of the a case where two building that are reconstructed as one.
                                }
                            }
                            \ffigbox[\FBwidth]{
                                \includegraphics[width=.45\textwidth, angle=270]{images/errors/facet/under_segmentation}
                            }{
                                \label{subfig::fus_2d}
                                \caption{
                                    Nadir projection of the erroneous building superposed on the corresponding orthoimage.
                                    This shows how a higher neighbooring building part can drive a misestimation of both facet planes which end up confused in one flat roof.
                                    The line segment highlighted in green gives away the fact that the roof was undersegmented.
                                }
                            }
                        \end{subfloatrow}
                    }{
                        \caption{
                            \label{fig::fus}
                            Illustration of a \gls{acr::fus} error.
                        }
                    }
                \end{figure}

                Usually automatic reconstruction methods rely on an initial surface (usually plane) extraction step that generates proposals for further refinement.
                Noise from stereo pairing or missing data in point clouds result in imprecisions in surface retrieval which then lead to surfaces being confused.
                The accuracy drops even further in some cases.
                For instance, superstructures like dormer windows can be big enough to be confused with the roof facets.
                An other setting where surfaces are hard to extract is when a building part is shadowed by an other one that is higher.
                This is depicted in Figure~\ref{fig::fus}.
                Methods relying only on plane extraction~\parencite{durupt2006automatic,taillandier2004automatic,nan2017polyfit} are particularly vulnerable to this error type.\\

                This defect can be mitigated through the use of \gls{acr::3d} lines as cues to guide the plane extraction~\parencite{zebedin2008fusion,sinha2009piecewise}.
                One can also discard plane extraction all together and try to reconstruct the building surface based only on \gls{acr::3d} lines (in other words, a wireframe building model)~\parencite{langlois2019surface,hofer2017efficient}.
                Using grammars of possible stuctures is another alternative, provided it is adequate to the modeled building.
                \textcite{lafarge2008structural} fits the best type of roof models to alleviate issues caused by high levels of noise like when working with Satellite based \glspl{acr::dsm}.

            \paragraph{\texttt{\acrlong*{acr::fos}}}
                \gls{acr::fos} corresponds to the case where two or more facets are modeled as one, as illustrated in Figure~\ref{fig::fos}.
                This is to \gls{acr::bos} what \gls{acr::fus} is to \gls{acr::bus}.\\

                \begin{figure}[H]
                    \centering
                    \ffigbox[\FBwidth]{
                        \begin{subfloatrow}[2]
                            \ffigbox[\FBwidth]{
                                \includegraphics[width=.45\textwidth]{example-image}
                            }{
                                \label{subfig::fos_3d}
                                \caption{\gls{acr::3d} depiction of the error.}
                            }
                            \ffigbox[\FBwidth]{
                                \includegraphics[width=.45\textwidth, angle=270]{images/errors/facet/over_segmentation}
                            }{
                                \label{subfig::fos_2d}
                                \caption{
                                    A slim chimney in the below corner of the ridge results in a defect ladden \gls{acr::dsm} which translates into oversegmentation.
                                    The erroneous edges are colored in red.
                                    We can check using the orthoimage that these are not real.
                                }
                            }
                        \end{subfloatrow}
                    }{
                        \caption{
                            \label{fig::fos}
                            Illustration of a \gls{acr::fos} error.
                        }
                    }
                \end{figure}

                As seen previously, lines are used to help find correct planes and avoid under segmentation.
                However, an overdetection of lines can lead to an oversegmentation of the model.
                This is not rare due to problems that can be encountered with illumination conditions: for instance, a roof structure can cast its shadow on a neighbooring one and cause a gradient in image signal that will be translated to a virtual edge.
                Superstructures play also a negative role just as explained previously.
                This time it is the ones that are small in planar size compared to the noise order of magnitude that are not detected but add bumps that polutes the signal and result in an misestimation of planes.
                High neighbooring buildings are also to blame due to the same reasons as with \gls{acr::fus}, but this time results in bumps like with superstructures.\\

                To solve this kind of issues, mesh simplification techniques can be helpful.
                In fact,~\textcite{verdie2015lod} uses this approach to smooth away these problems and produce a good generalization of the underlying buildings.
                Another way is to filter the extracted lines relying on redundancy as shown in~\textcite{michelin2013quality}.
                Grammar based methods can equally come to rescue.
                As an example,~\textcite{bredif20073d} uses a set parameteric models in order to model superstructures and better fit \gls{acr::lod}-2 roof facets.

            \paragraph{\texttt{\acrlong*{acr::fib}}}
                \gls{acr::fib} corresponds to the case where at least one facet border is incorrectly located.
                As en example, Figure~\ref{fig::fib} shows that the central edge that links the two main roof sides does not correspond to the one on the image position.
                This is a purely geometrical error similarly to \gls{acr::bib}.\\

                \begin{figure}[H]
                    \centering
                    \ffigbox[\FBwidth]{
                        \begin{subfloatrow}[2]
                            \ffigbox[\FBwidth]{
                                \includegraphics[width=.45\textwidth]{example-image}
                            }{
                                \label{subfig::fib_3d}
                                \caption{\gls{acr::3d} depiction of the error.}
                            }
                            \ffigbox[\FBwidth]{
                                \includegraphics[width=.45\textwidth]{images/errors/facet/border}
                            }{
                                \label{subfig::fib_2d}
                                \caption{
                                    The nadir projection of the model on the orthoimage provides the real location (in green) of the edge.
                                }
                            }
                        \end{subfloatrow}
                    }{
                        \caption{
                            \label{fig::fib}
                            Illustration of a \gls{acr::fib} error.
                        }
                    }
                \end{figure}

                Line extraction is usually very faithfull to the data and depends mostly to the resolution and quality of the input data used for modeling.
                The most likely reason usually behind this kind of errors are usually imprecise fitting of primitives that lead to a shifted intersecting edge such as shown in Subfigure~\ref{subfig::fib_2d}.\\

                Just as with oversegmentation, one way to make lines retrieval more accurate is to rely on redundancy by extracting them from different modalities.
                An alternative is to rely on symetries \addref to automatically correct surface intersections.

            \paragraph{\texttt{\acrlong*{acr::fit}}}

                \gls{acr::fit} corresponds to the case where the facet suffers from topological defects such as wrong primitive fitting (for example, a dome approximated by planar polygons).
                In Figure~\ref{fig::fit}, we can observe how two cylindrical towers were reconstructed as a rectangular parallelepiped.\\

                \begin{figure}[H]
                    \centering
                    \ffigbox[\FBwidth]{
                        \begin{subfloatrow}[2]
                            \ffigbox[\FBwidth]{
                                \includegraphics[width=.45\textwidth]{example-image}
                            }{
                                \label{subfig::fit_3d}
                                \caption{\gls{acr::3d} depiction of the error.}
                            }
                            \ffigbox[\FBwidth]{
                                \includegraphics[width=.45\textwidth]{images/errors/facet/topology}
                            }{
                                \label{subfig::fit_2d}
                                \caption{
                                    the nadir projection reveals the true form (in green) of the towers that were completly misrepresented (in red).
                                }
                            }
                        \end{subfloatrow}
                    }{
                        \caption{
                            \label{fig::fit}
                            Illustration of a \gls{acr::fit} error.
                        }
                    }
                \end{figure}

                This can be due to various reasons.
                Most methods rely on the assumption that buildings are piecewise linear~\parencite{nan2017polyfit} or Manhattan-world like~\parencite{li2016boxfitting}.
                This is evidently not always the case (\textit{cf.} Subfigure~\ref{subfig::fit_2d}).
                Even with the right assumptions, this depicted case cannot have been well modeled.
                If so it would have at least approximated the circular cylindrical structures with a regular polygon cylinder.
                This is in fact due to the fact that the quality of the data was poor and was unreliable as was explained with the \gls{acr::fus} case in Subfigure~\ref{subfig::fus_2d}.
                The same effect can cause a missing hole being undetected.\\

                Solving this issue is hard besides the obvious change of primitive assumptions.
                It depends highly on the quality of the data.
                One can try to go overcome this issue once again, like with the undersegmentation problem, thanks to line detections to reveal convoluted structures when relying on plane extraction only. 

            \paragraph{\texttt{\acrlong*{acr::fig}}}
                \gls{acr::fig} corresponds to the case of inaccurate facet geometric estimation.
                In mathematical terms, this means that the surface primitive parameters where misestimated.
                In Figure~\ref{fig::fig}, the planar surface slope was miscalculated as flat while it was of \textit{ca.} \SI{25}{\degree}.\\

                \begin{figure}[H]
                    \centering
                    \ffigbox[\FBwidth]{
                        \begin{subfloatrow}[2]
                            \ffigbox[\FBwidth]{
                                \includegraphics[width=.45\textwidth]{example-image}
                            }{
                                \label{subfig::fig_3d}
                                \caption{\gls{acr::3d} depiction of the error.}
                            }
                            \ffigbox[\FBwidth]{
                                \includegraphics[width=.45\textwidth]{images/errors/facet/geometry}
                            }{
                                \label{subfig::fig_2d}
                                \caption{
                                    This kind of error is impossible is vizualize on the orthoimage.
                                    Projected on the \gls{acr::dsm}, we can reveal the sloped character of the face.
                                }
                            }
                        \end{subfloatrow}
                    }{
                        \caption{
                            \label{fig::fig}
                            Illustration of a \gls{acr::fig} error.
                        }
                    }
                \end{figure}

                This is linked in particular to the input sensor data quality.
                The case of Subfigure~\ref{subfig::fig_2d} illustrates how neighbooring building parts polutes the data as it blunts away the slope of the roof.
                Semantics play again a role in detecting a purely geometric error.
                For correction, as the corruption in the data resulting from the modeling limits could be filtered out using semantics, one can reestimate the parameters of the primitives.
                Failing that, this correction step is usually conducted by human operators.

    \subsection{\textsc{Discussion}}
        \label{subsec::semantic_evaluation::overhead::discussion}

        In the previous subsection, we defined \texttt{atomic} errors based on our observations related to automatic overhead modeling of urban scenes.
        We also discussed how each type of defects can be caused and how to be fixed.
        Herein, we explore the properties of the ensuing taxonomy of errors and how they came to be.
        In addition, we discuss also how this taxonomy is related to the ones from the litterature.

        \subsubsection{\textsc{Error taxonomy properties}}
            Auto-similarity.
        
        \subsubsection{\textsc{Related taxonomies}}
            These errors can be related to state-of-the-art labels.
            For instance, ``Missing Node'' (\textit{resp.} ``False Node'', ``Missing Edge'' and ``False Edge'') in~\textcite{xiong2014graph} correspond to, or are included in, the topological \textit{atomic} errors from the \textit{Facet Errors} family: \texttt{FUS} (\textit{resp.} \texttt{FOS}, \texttt{FIT}, and \texttt{FIT}).
            The difference is that we distinguish flaws that can affect superstructure facets (\gls{acr::lod}-3) from the ones that impair building facets (\gls{acr::lod}-2).
            The taxonomy developed by~\textcite{michelin2013quality}, on the other hand, is closer to ours.
            In fact, while footprint errors is reshuffled into \textit{Building Errors} as \texttt{BIB} (``erroneous outline'' and ``imprecise footprint'') and \texttt{BIT} (``missing inner court'') , intrinsic reconstruction errors (``over-segmentation'', ``under segmentation'', ``inexact roof'' and ``Z translation'') can be re-adapted into both family errors.
            Finally, ``vegetation occlusion'' and `` non existing'' are gathered into the \texttt{Unqualifiable} label at \texttt{finesse} level 0.~\textcite{boudet2006supervised}, however, studied the acceptability of a model in a whole.
            Their taxonomy cannot directly fit with our labels.
            The acceptability dimension can be incorporated into our framework by attributing an confidence score to each error: for example, a prediction probability.        

\section{\textsc{Parametric label extraction}}
        \label{sec::semantic_evaluation::label_extraction}
    At evaluation time, three parameters play a role in determining which error labels to consider.
    The first is the \textbf{\gls{acr::elod}}. Every reconstruction method targets a certain set of \glspl{acr::lod}.
    In consequence, when assessing a reconstruction, a \gls{acr::lod} must be specified. At a given \gls{acr::elod}, all error families involving higher orders will be ignored.
    Depending on the target of the qualification process, a \textbf{finesse} level might be preferred.
    This second evaluation parameter specifies the appropriate semantic level at which errors will be reported.
    The last one is error \textbf{exclusivity}. It conveys family error hierarchy. Errors of a given \gls{acr::lod}$ = l$ family are prioritized over ones with higher \gls{acr::lod}$ > l$.

    \subsection{\textsc{Evaluation parameters}}
    \subsection{\textsc{Evaluation labels}}
