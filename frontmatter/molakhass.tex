اصّناعة ديال انّمادج تولاتية لبعاد ديال لبنايات بطريقة تلقاءيّة ولّات معمول بيها هاد ليّام.
كاين فهاد عشرين عام الّخرّة عدد كافي من لمراجع لّي اتّطرقات لهاد لموضوع، زيد عليها عدد ديال لبارامج.
لموشكيلة أنّاهو لمجالات لحضريّة منوّعة بزّاف، اشّي الّي تيخلّي أنّا، بطريقة ولّا اخرة، أين موقاربة تقدر تغلط.
هادشي تيدفع فلواقع أنّانا نتءكّدو باشّوف من اصّحة ديال انّمادج، لقاضيّة الّي تاتاخد عاداتان 2 سوايع فكيلومطرمربع لكلّا بناية لكلّا خبير.
على ود هادشي تانقتارحو طريقة باش نقيّمو انّمادج ديال لبنايات تلقاءيّان.
أول حاجة هيّا أنّانا صنّفنا لأغلاط الّي تقدر تخسّر اّنمادج بلوحدات أو بشكل هرميّ.
هاد اتّفراز تيفرّق لأول مرّة مابين لمشاكيل ديال اشّكل من لأغلاط فادّقة.
باش نقشعو هاد لأغلاط الّي عرّفنا، تانعوّلو على لخاصّيات لهندسية ديال انّمودج الّي تانتحقّقومنّو، بازيّادة على خاصيّات برّانيّة منّي تانقارنو انّمودج مع تصويرة عموديّة ولّا تصويرة ديال لعمق.
فهاد لخدمة الّي درنا، قتارحنا أولا خاصيّات بصيطة أوليّة باش نتءكدّو من لمقاربة ديالنا.
هاد لخاصيّات تاندوزوهوم لموصنّيفات بحال لمفرّز بأوسع هامش أولّا لغابة ديال اتّفراق لعلّا وي.
فلمحالة اتّالية، قتارحنا خاصيّات آخرة باش نتحقّقو من كولّا نمودج.
هاد لخاصيّات تيعوّلو على نفس لمعطايات فادّخلة، والاكن تاتعتمد علا طريقات مقدّمة كتر.
توكدنا من هاد اتّسلسل باتّجربة على تلاتة ديال لمدينات ففرانسا: \textbf{ايلونكور} (2009 بناية)، \textbf{نانط} (748 بناية) أو لحيّ 13 ديال باريز (\textbf{باريز-13}: 478 بناية).
انّتيجة ديال هاد اتّجاريب عطات معدّل ديال ادّقة أو لكموليّة مابين \SI{80}{\percent} أو \SI{99}{\percent}.
تأكدنا تاني فهاد اتّجاريب أنّا اتّفراز الّي تانتعلموه فمنطقة ايقدر ايتطبق لمنطقات أوخرى بسيفتو قابل إيتحوّل أو إيتعمّم أو إيتدرّج.\\

\textbf{الكلمات الرئيسية.~\ ~\ } انّمدجة تولاتيّة لبعاد ديال لمناطق لحضريّة، مجموعة دلموعطايات، اتّحقّق ديال لجودة، اتّصنيف ديال لأغلاط، لقشيع دلأغلاط، اتّصاور اسّماويّة، اتّصاور ارّقميّة ديال اصّطح، لهندسة، اتّعلّوم بلإيحصاء، اتّصنيف متعدّد.
