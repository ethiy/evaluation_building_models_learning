يعد الإنشاء التلقائي لنماذج بناء  من البيانات الجغرافية المكانية إجراءً قياسيًا الآن.
تغطي الأدبيات الوفيرة العقدين الأخيرين والعديد من البرامج المتاحة الآن.
ومع ذلك ، فإن المناطق الحضرية هي بيئات معقدة للغاية.
حتمًا ، لا يزال يتعين على الممارسين إجراء تقييم بصري ، على مستوى المدينة ، لصحة هذه النماذج واكتشاف أخطاء إعادة البناء المتكررة.
تعتمد هذه العملية على الخبراء وتستهلك الكثير من الوقت مع حوالي  {\ ساعة \ لكل \ كم \ مربع \ لكل \ خبير}.
يقترح هذا العمل نهجًا لتقييم جودة نماذج البناء  تلقائيًا.
يتم تجميع الأخطاء المحتملة في تصنيف هرمي ونموذجي جديد.
وهذا يسمح ، لأول مرة ، بالتعامل مع أخطاء الدقة والنمذجة ، بغض النظر عن مستوى تفاصيل المباني النموذجية.
يتم توقع جودة النماذج باستخدام الخصائص الهندسية للمباني ، وعند توفرها ، صور عالية الدقة ونماذج سطح رقمية.
يتم إدخال الخط الأساسي للميزات اليدوية ، ولكن العامة ، في مصنفات rf أو svm.
تم اقتراح ميزات متقدمة ، تعتمد على نواة الرسم البياني وكذلك شبكات التشتت ، لتأخذ في الاعتبار بشكل أفضل الهيكل.
يتم دراسة كل من الحالات متعددة الفئات والعلامات المتعددة: نظرًا للترابط بين فئات الأخطاء ، من الممكن استرداد جميع الأخطاء في نفس الوقت مع توقع المباني الصحيحة والخاطئة.
تم اختبار الإطار المقترح على ثلاث مناطق حضرية متميزة في فرنسا مع أكثر من 3000 مبنى.
يتم تحقيق قيم 80 99 للأخطاء الأكثر تكرارًا.
ولأغراض قابلية التوسع ، تمت دراسة تأثير تكوين المنطقة الحضرية على التنبؤ بالأخطاء ، من حيث إمكانية النقل والتعميم وتمثيل المصنفات.
ويبين ضرورة بيانات الاستشعار عن بعد متعددة الوسائط ومزج عينات التدريب من مدن مختلفة لضمان استقرار نسب الكشف ، حتى مع أحجام مجموعات التدريب محدودة للغاية. \\

\textbf{الكلمات الرئيسية.} ~ \ ~  النمذجة الحضرية ، المباني ، مجموعة البيانات ، تقييم الجودة ، تصنيف الأخطاء ، اكتشاف الأخطاء ، الصور الجوية ، vhr ، dsm ، علم الهندسة ، التعلم الإحصائي ، التصنيف متعدد التصنيفات. \\
