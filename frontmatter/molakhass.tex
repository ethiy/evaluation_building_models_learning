\begin{RLtext}
    \begin{flushright}
        \vspace{-1.5cm}
        \spreadline{تعد أسواق الفائدة من أهم الأسواق المالية في العالم وذلك لكبر الحجوم المتبادلة بها و لمدى}
        \vspace{-1cm}
        \spreadline{أثرها على الاقتصاد. وتعتبر الصكوك من بين أهم الأصول المالية المتبادلة في أسواق الفائدة. }
        \vspace{-1cm}
        \spreadline{عندما تصدر الصكوك في السوق، فإنها تكون رهينة لخطرين أساسيين و هما خطر أسعار }
        \vspace{-1cm}
        \spreadline{الفائدة و المرتبط بتقلبات أسعار الفائدة بالسوق، و خطر الائتمان المرتبط بالجدارة الائتمانية }
        \vspace{-1cm}
        \spreadline{للطرف المقابل. و في هذا السياق، يأتي العمل المقدم في هذا التقرير لتلبية حاجة التاجر }
        \vspace{-1cm}
        \spreadline{بالصكوك و المتمثلة في حماية الصكوك الممتلكة ضد مخاطر أسعار الفائدة و مخاطر الائتمان }
        التي قد تتعرض لها.
        
        \medskip
        
        \spreadline{يهدف الجزء الأول من هذا العمل إلى تسعير الصكوك حسب المبادئ المتفق عليها. سنتطرق}
        \vspace{-1cm}
        \spreadline{بعدها إلى مخاطر أسعار الفائدة التي قد تتعرض لها الصكوك و نشرح كيف يمكن حمايتها}
        \vspace{-1cm}
        \spreadline{من هذا الخطر باستعمال عقدين اشتقاقيين و هما مبادلة سعر الفائدة و العقود الآجلة.}
        \vspace{-1cm}
        \spreadline{في الجزء الثاني من التقرير، سنشرح بتفصيل مبادلة سعر الفائدة ثم سنستعملها في وضع}
        \vspace{-1cm}
        \spreadline{استراتيجية لتغطية خطر سعر الفائدة. ثم في الجزء الثالث، سنتعرض إلى العقود الآجلة}
        \vspace{-1cm}
        \spreadline{بتدقيق، و نقدم مختلف الاستراتيجيات الموضوعة و المرتكزة عليها من أجل حماية الصكوك}
        \vspace{-1cm}
        \spreadline{من خطر سعر الفائدة. و في جزء أخير، سنهتم بشرح خطر الائتمان الذي قد يتعرض إليه}
        \vspace{-1cm}
        \spreadline{مالك الصكوك ثم سنقدم استراتيجية ترتكز على مقايضة التخلف عن سداد الائتمان للتغطية}
        من هذا الخطر.
    \end{flushright}
\end{RLtext}