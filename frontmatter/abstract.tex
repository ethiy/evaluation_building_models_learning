The automatic generation of 3D building models from geospatial data is now a standard procedure.
An abundant literature covers the last two decades and several softwares are now available.
However, urban areas are very complex environments.
Inevitably, practitioners still have to visually assess, at city-scale, the correctness of these models and detect frequent reconstruction errors.
Such a process relies on experts, and is highly time-consuming with approximately \SI[per-mode=repeated-symbol]{2}{\hour\per\km\squared\per\expert}.
This work proposes an approach for automatically evaluating the quality of 3D building models.
Potential errors are compiled in a novel hierarchical and modular taxonomy.
This allows, for the first time, to disentangle fidelity and modeling errors, whatever the level of details of the modeled buildings.
The quality of models is predicted using the geometric properties of buildings and, when available, Very High Resolution images and Digital Surface Models.
A baseline of handcrafted, yet generic, features is fed into a \acrlong*{acr::rf} or \acrlong*{acr::svm} classifiers.
Advanced features, relying on graph kernels as well as Scattering Networks, were proposed to better take into consideration structure.
Both multi-class and multi-label cases are studied: due to the interdependence between classes of errors, it is possible to retrieve all errors at the same time while simply predicting correct and erroneous buildings.
The proposed framework was tested on three distinct urban areas in France with more than 3,000 buildings.
\SIrange{80}{99}{\percent} F-score values are attained for the most frequent errors.
For scalability purposes, the impact of the urban area composition on the error prediction was also studied, in terms of transferability, generalization, and representativeness of the classifiers.
It shows the necessity of multi-modal remote sensing data and mixing training samples from various cities to ensure a stability of the detection ratios, even with very limited training set sizes.\\

\textbf{Keywords.}~~\gls*{acr::3d} urban modeling, Buildings, Dataset, Quality assessment, Error taxonomy, Error detection, Aerial imagery, \acrlong*{acr::vhr}, \acrlong*{acr::dsm}, Geometry, Statistical learning, Multi-label classification.\\
