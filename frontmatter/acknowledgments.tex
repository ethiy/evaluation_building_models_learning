~\\

I would like to thank Clément who not only was my advisor but also supervised my work from the start.
I learned a lot with him in different aspects.
He always pushed me to do better and be as rigorous as humanly possible.
In fact, despite being very knowledgeable, he is always ready to learn new things.

I have to acknowledge equally Arnaud (a.k.a. \textit{Chuck}) for always being available.
He seems to always have answers to every question, be it scientific, technical, administrative or historical of nature: hence the nickname.
He was also available, to his great sorrow, to suffer a great deal trying to decipher the first iterations of my writings.

Florent deserves also a special recognition.
Despite being far from us, he provided us with sound advice at each turn.
I would like to thank him and Pierre \textsc{Alliez} for inviting me for a stay at the Titane team.
I thank Pierre also for being part of the thesis committee.\\

I would like to address my recognition to Renaud \textsc{Marlet}, who was interested from the beginning in our work.
Being a member of the thesis committee, he also accepted to preside over the defense jury.
I thank also Franz \textsc{Rottensteiner} and Gilles \textsc{Gesquière} for their reviews and thorough comments.
I have George \textsc{Vosselman} and Pooran \textsc{Memari} to thank for their relevant remarks.\\

I would like to acknowledge Mohammed \textsc{El Rhabi} who was of great help as my academic supervisor at \'Ecole des Ponts Paristech.\\

I thank also Sébastien who was, for the first half of my stay at LaSTIG, my office mate.
He had to suffer my (terrible) juggling skills and my numerous \verb!QGIS! questions.
After suffering for one year and a half from the scorching heat of the south wing, I moved north to find myself with Teng, Raphaël and Stéphane for the rest of my stay.
I have to thank them for the time we spent together, mainly playing football in the office (sorry Teng) and learning each other's languages.
Stéphane deserves a special mention, as he was of great help especially during the last months of my thesis.\\

Je tiens à remercier en particulier David Correia qui nous a énormément aidé, surtout quand on commettait des anneries (\verb!<>:/usr/lib/foo$ sudo rm -rf ../!).
Je ne peux pas aussi oublier Laurent Schneider qui m'a été d'une grande aide pour déchiffrer les erreurs de compilation de \verb!CGAL! et qui a joué beaucoup, malgré lui, le rôle de rubber duck.
Marie-Claude ainsi que Alain qui, à eux seuls, maintiennent le fonctionnement du laboratoire, méritent la reconnaissance de tout le personnel du LaSTIG.\\

During my three year stay at LaSTIG, I had the opportunity to meet some endearing people out of which I can mention only a few: Nathan, Bastien, \foreignlanguage{arabic}{عمّي علي}, Quy Thy, Yilin (always making jokes), \foreignlanguage{arabic}{محمّد}, \foreignlanguage{arabic}{أمين} (Diabetes guy), Imran (SJW), Lâman (Tea guy), Anatol (Némar fan boy), Laurence (standing tall), Mattia (the literary guy who uses Haskell), Ewelina (globe trotter), Laurent (l'élite), \foreignlanguage{arabic}{إيمان}, Jean-Pierre, Quoc, Neelanjan, Vivien (\foreignlanguage{greek}{για χαρά}), David (le Nantais), Bahman, Clément ``Junior'', Yann (le faux breton), Alexandre (a.k.a. Waldo), Paul (Coffee dictator), Maxime (le banlieusard de Lyon), Evelyne, Qasem (a.k.a. Alex 2), Yanis (l'autre banlieusard de Lyon), Bruno, Mathieu, Guillaume, Julien (a.k.a. Jupé), Marc (spécialiste de l'apéro), Bénedicte and others.

Je tiens aussi à remercier Marc Poupée, et l'ensemble du staff de l'\acrshort*{acr::ensg}, avec qui j'ai collaboré pour dispenser les cours au cycle ingénieur.\\

I would also like to thank members of the Titane team during my stay at Inria Sophia Antipolis: namely Cédric, Muxingzi, Gaétan, Nicolas, Jean-Philippe and Flora.\\

\selectlanguage{arabic}
ضاروري نشكر أهم ناس فحياتي.
تانهضر بلخصوص على واليديّا، الّي ضحّاو بكترمن 25 عام من شبابهوم على ودّ موستقبالي.
بغيت نشكر تاني هدى، خطيبتي، الّي صبرات معايا هاد تلت سنين أو تحملّات جزء من اضّغط ديال خدمتي.
مروان ’’ضارك فزّ''، محمّد ’’اصّايمو''، يوسف ’’الّي عارف راسو'' أو سامي ’’يا زلمي'' تيستاحقو حتّى هوما اشّكر أو لعتيراف ديالي: وقفو معيا فواحد من أصعب لمواقيف فحياتي.
\selectlanguage{english}
