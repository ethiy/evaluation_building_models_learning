\documentclass[tikz]{standalone}

\usepackage[T1]{fontenc}
\usepackage[english]{babel}
\usepackage{fourier}

\usepackage{xcolor}
\definecolor{darkgreen}{RGB}{0, 155, 85}

\usepackage{standalone}

\usepackage{pgfplots}

\usepackage{tikz}
\usetikzlibrary{shapes, arrows, shadows, calc, decorations, 3d}

\usepackage{mathrsfs, amsmath}


\begin{document}
    \begin{tikzpicture}
        \node (facet_graph) {\includestandalone[mode=buildnew, height=4cm]{figures/graphical_abstract/building_graph}};
        \path (facet_graph.south west) node[visible on=<5->, anchor=north west, align=left] (multi_scale_laplacian_kernel) {\footnotesize \textcolor{IGNGreen}{\(\blacktriangleright\)} Multiscale Laplacian kernel;};
        \path (multi_scale_laplacian_kernel.south west) node[visible on=<6->, anchor=north west, align=left] (graph_hopper_kernel) {\footnotesize \textcolor{IGNGreen}{\(\blacktriangleright\)} Graph hopper kernel;};
        \path (graph_hopper_kernel.south west) node[visible on=<7->, anchor=north west, align=left] (propagation_kernel) {\footnotesize \textcolor{IGNGreen}{\(\blacktriangleright\)} Propagation kernel.};

        \path (facet_graph.south west) node[visible on=<2-3>, anchor=north west, align=left] (random_walk_kernel) {\footnotesize \textcolor{IGNGreen}{\(\blacktriangleright\)} Random walk kernel;};
        \path (random_walk_kernel.south west) node[visible on=<3>, anchor=north west, align=left] (svm_kernel) {\footnotesize \textcolor{IGNGreen}{\(\blacktriangleright\)} SVM \(\vartheta\) kernel.};

        \path (facet_graph) + (-.9, -1) node[blue, visible on=<4-7>] (degree_area_circumference) {
            \tiny \(\begin{bmatrix}
                \operatorname{degree}(f)\\
                \operatorname{Area}(f)\\
                \operatorname{Circumference}(f)
            \end{bmatrix}\)
        };
        \path (degree_area_circumference) node[blue, visible on=<8>] (centroid) {
            \scriptsize \(\operatorname{Centroid}(f)\)
        };
        \path (degree_area_circumference) node[blue, visible on=<9>] (normal) {
            \scriptsize \(\operatorname{normal}(f)\)
        };
    \end{tikzpicture}
\end{document}
